\documentclass[a4paper]{report}
\usepackage{float}
\usepackage[utf8]{inputenc}
\usepackage[italian]{babel}
\usepackage{amsmath}
\usepackage{amssymb}
\usepackage{graphicx}
\graphicspath{ {./graph/} }
\usepackage{lmodern}
\usepackage{kpfonts}
\usepackage{titlesec}
\usepackage{listings}
\usepackage{color}
\usepackage{fontspec}
\usepackage{multirow}
\usepackage{float}
\usepackage{array}
\usepackage{afterpage}

\usepackage[font={small}, labelfont={bf}, format=hang, skip=8pt]{caption}



 \setmainfont{Helvetica} % Set the main font to Helvetica
 \setmonofont{Noto Sans Mono} % Set the monofont to Andale Mono

% Imposta lo spacing tra il titolo del paragrafo e il testo successivo
% \usepackage[left=3.6cm,right=3.6cm,top=2.5cm,bottom=2.25cm]{geometry}

\definecolor{grigio}{rgb}{0.95,0.95,0.95}
\definecolor{mygrey}{rgb}{0.8,0.8,0.8}
\definecolor{mygreen}{rgb}{0.2,0.4,0.2}

\lstset{
    firstnumber=1,                % start line enumeration with line 1000
    language=Verilog,
    numbers=left,
    stepnumber=1,
    numbersep=4pt,
    numberstyle=\tiny\color{mygrey}, % the style that is used for the line-numbers
    backgroundcolor=\color{grigio},
    showspaces=false,
    showstringspaces=false,
    showtabs=false,
    tabsize=3,
    captionpos=b,
    breaklines=true,
    breakatwhitespace=true,
    escapeinside={\%*}{*)},
    basicstyle=\ttfamily\fontsize{8pt}{10pt}\selectfont, % Set the basic code style to a smaller font
    keywordstyle=\bfseries,
    commentstyle=\color{mygreen}}

\titleformat{\chapter}[display]
{\normalfont\huge\bfseries\fontsize{14pt}{10pt}\selectfont}{\chaptertitlename\ \thechapter}{18pt}{\Huge}

\author{Tommi Bimbato VR500751, Antonio Iovine VR504083}
\title{Elaborato Assembly \\ \normalsize Corso di Architettura degli Elaboratori A.A. 2023/2024 \\ Prof.\ Franco Fummi, Prof.\ Michele Lora}


\begin{document}

\begin{titlepage}
    \maketitle
\end{titlepage}

% Applica il nuovo stile di intestazione a tutte le pagine tranne la prima
\thispagestyle{empty} % Rimuove l'intestazione dalla pagina del titolo

\tableofcontents % Aggiunge una tabella dei contenuti

\begin{abstract}

  L'obiettivo del progetto è sviluppare un software per la pianificazione delle attività di un sistema produttivo.
  La produzione è organizzata in slot temporali uniformi, durante i quali un solo prodotto può essere in fase di lavorazione. 
  Il software consentirà di ottimizzare la pianificazione delle attività secondo due algoritmi di pianificazioni differenti.
  L'intero software è stato sviluppato in linguaggio Assembly e testato su un insieme di dati di prova allegati a questa documentazione.

\end{abstract}

\chapter{Introduzione}
\section{Approccio progettuale}
L'elaborato è stato condotto seguendo un approccio metodologico strutturato.
Inizialmente, è stata eseguita un'analisi dettagliata per identificare i requisiti
e le funzionalità principali del software. Questo processo ha consentito
una comprensione completa del contesto operativo e degli obiettivi da raggiungere.

Successivamente, è stata sviluppata una bozza del software utilizzando il linguaggio
di programmazione C. Questo ha permesso la traduzione dei requisiti in
una struttura logica e l'identificazione dell'architettura generale del software.

Parallelamente, sono stati definiti gli spazi di memoria necessari
per la memorizzazione dei dati durante l'esecuzione del programma.
Ciò ha garantito un utilizzo efficiente delle risorse disponibili.

Successivamente, è stata sviluppata una stesura iniziale del codice assembly,
ottimizzando l'efficienza e la velocità di esecuzione del software.

Infine, sono stati eseguiti test approfonditi per verificare il corretto
funzionamento del programma e identificare eventuali aree di miglioramento.
L'iterazione su questo processo ha portato a modifiche e ottimizzazioni
fino al raggiungimento di un livello soddisfacente di prestazioni e funzionalità.

\section{Analisi delle specifiche}
La produzione è organizzata in slot temporali uniformi, durante i quali un solo prodotto può essere in fase di lavorazione.
Il programma analizza una serie di prodotti, ognuno caratterizzato da un identificativo, una durata, una scadenza e una priorità secondo le specifiche seguenti:
\begin{itemize}
    \item Identificativo: un codice da 1 a 127;
    \item Durata: il numero di slot temporali per il completamento (da 1 a 10);
    \item Scadenza: il limite massimo di tempo entro cui il prodotto deve essere completato (da 1 a 100);
    \item Priorità: un valore da 1 a 5, che indica sia la priorità che la penalità per il ritardo sulla scadenza\footnote{Il valore 5 indica la priorità più alta.}.
\end{itemize}
Per ogni prodotto consegnato in ritardo, l'azienda deve pagare una penale, calcolata moltiplicando la priorità del prodotto per il ritardo in unità di tempo rispetto alla scadenza.

\subsection{Input}
Quando l'utente fornisce due file come parametri da linea di comando, il primo viene considerato come l'input, mentre il secondo viene utilizzato per salvare i risultati della pianificazione. Ad esempio:

\begin{verbatim}
pianificatore Ordini.txt Pianificazione.txt
\end{verbatim}

In questo caso, il programma caricherà gli ordini dal file \texttt{Ordini.txt} e salverà le statistiche stampate a video nel file \texttt{Pianificazione.txt}.

Se l'utente fornisce solo un parametro, la stampa su file verrà ignorata.

Il file degli ordini dovrà avere un prodotto per riga, con tutti i parametri separati da virgola. Ad esempio, se l'rodine fosse:
\begin{verbatim}
ID: 4; Durata: 10; Scadenza: 12; Priorità: 4;
\end{verbatim}
Il file dovrà contenere la seguente riga:
\begin{verbatim}
4,10,12,4
\end{verbatim}

Una volta letto il file, il programma mostrerà il menu principale che chiede all'utente quale algoritmo di pianificazione desidera utilizzare.



\subsection{Algoritmi di pianificazione}
Una volta letto il file, il programma visualizzerà il menu principale, permettendo all'utente di selezionare l'algoritmo di pianificazione desiderato. Le opzioni disponibili sono:

\begin{enumerate}
    \item Earliest Deadline First (EDF): Si pianificano per primi i prodotti con scadenza più vicina. In caso di parità nella scadenza, si considera la priorità più alta.
    \item Highest Priority First (HPF): Si pianificano per primi i prodotti con la priorità più alta. In caso di parità di priorità, si considera la scadenza più vicina.
\end{enumerate}

L'utente può selezionare uno dei due algoritmi per la pianificazione delle attività del sistema produttivo.

\subsection{Output}

Pippo puppu bau bau ciao ciao come va! hello its me i was wondering if after all these years you'd like to meet to go over everything they say that time's supposed to heal ya but i ain't done much healing...

\section{Codice}

Hello its me i was wondering.. if after all these years you'd like to meet... to go over everything... they say that time's supposed to heal ya... but i ain't done much healing...

\begin{lstlisting}[firstnumber=34]
  Codice in ASM!!!
  int main pippo VOID !!!
  ; Gibberish Assembly Code
  mov eax, 0
  add ebx, eax
  sub ecx, ebx
  mul edx, ecx
  div esi, edx
  push eax
  pop ebx
  inc ecx
  dec edx
  jmp label
  label:
  nop
  cmp eax, ebx
  jne label

\end{lstlisting}


\vspace{20pt}

\end{document}
