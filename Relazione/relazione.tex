\documentclass[a4paper]{report}
\usepackage{float}
\usepackage[utf8]{inputenc}
\usepackage[italian]{babel}
\usepackage{amsmath}
\usepackage{amssymb}
\usepackage{graphicx}
\graphicspath{ {./graph/} }
\usepackage{lmodern}
\usepackage{kpfonts}
\usepackage{titlesec}
\usepackage{listings}
\usepackage{color}
\usepackage{fontspec}
\usepackage{multirow}
\usepackage{float}
\usepackage{array}
\usepackage{afterpage}

\usepackage[font={small}, labelfont={bf}, format=hang, skip=8pt]{caption}



% \setmainfont{Helvetica} % Set the main font to Helvetica
% \setmonofont{Noto Sans Mono} % Set the monofont to Andale Mono

% Imposta lo spacing tra il titolo del paragrafo e il testo successivo
\usepackage[left=3.6cm,right=3.6cm,top=2.5cm,bottom=2.25cm]{geometry}

\definecolor{grigio}{rgb}{0.95,0.95,0.95}
\definecolor{mygrey}{rgb}{0.8,0.8,0.8}
\definecolor{mygreen}{rgb}{0.2,0.4,0.2}

\lstset{
    firstnumber=1,                % start line enumeration with line 1000
    language=Verilog,
    numbers=left,
    stepnumber=1,
    numbersep=4pt,
    numberstyle=\tiny\color{mygrey}, % the style that is used for the line-numbers
    backgroundcolor=\color{grigio},
    showspaces=false,
    showstringspaces=false,
    showtabs=false,
    tabsize=3,
    captionpos=b,
    breaklines=true,
    breakatwhitespace=true,
    escapeinside={\%*}{*)},
    basicstyle=\ttfamily\fontsize{8pt}{10pt}\selectfont, % Set the basic code style to a smaller font
    keywordstyle=\bfseries,
    commentstyle=\color{mygreen}}

\titleformat{\chapter}[display]
{\normalfont\huge\bfseries\fontsize{14pt}{10pt}\selectfont}{\chaptertitlename\ \thechapter}{18pt}{\Huge}

\author{Tommi Bimbato VR500751, Antonio Iovine VR504083}
\title{Elaborato Assembly \\ \normalsize Corso di Architettura degli Elaboratori A.A. 2023/2024 \\ Prof.\ Franco Fummi, Prof.\ Michele Lora}


\begin{document}

\begin{titlepage}
    \maketitle
\end{titlepage}

% Applica il nuovo stile di intestazione a tutte le pagine tranne la prima
\thispagestyle{empty} % Rimuove l'intestazione dalla pagina del titolo

\tableofcontents % Aggiunge una tabella dei contenuti

\begin{abstract}

  L'obiettivo del progetto è sviluppare un software per la pianificazione delle attività di un sistema produttivo.
  La produzione è organizzata in slot temporali uniformi, durante i quali un solo prodotto può essere in fase di lavorazione. 
  Il software consentirà di ottimizzare la pianificazione delle attività secondo due algoritmi di pianificazioni differenti.
  L'intero software è stato sviluppato in linguaggio Assembly e testato su un insieme di dati di prova allegati a questa documentazione.

\end{abstract}

\chapter{Introduzione}
\section{Approccio progettuale}
L'elaborato è stato condotto seguendo un approccio metodologico strutturato.
Inizialmente, è stata eseguita un'analisi dettagliata per identificare i requisiti
e le funzionalità principali del software. Questo processo ha consentito
una comprensione completa del contesto operativo e degli obiettivi da raggiungere.

Successivamente, è stata sviluppata una bozza del software utilizzando il linguaggio
di programmazione C. Questo ha permesso la traduzione dei requisiti in
una struttura logica e l'identificazione dell'architettura generale del software.

Parallelamente, sono stati definiti gli spazi di memoria necessari
per la memorizzazione dei dati durante l'esecuzione del programma.
Ciò ha garantito un utilizzo efficiente delle risorse disponibili.

Successivamente, è stata sviluppata una stesura iniziale del codice assembly,
ottimizzando l'efficienza e la velocità di esecuzione del software.

Infine, sono stati eseguiti test approfonditi per verificare il corretto
funzionamento del programma e identificare eventuali aree di miglioramento.
L'iterazione su questo processo ha portato a modifiche e ottimizzazioni
fino al raggiungimento di un livello soddisfacente di prestazioni e funzionalità.

\section{Analisi delle specifiche}
\subsection{pippo}


\section{Codice}

Il Datapath è il componente che si occupa di elaborare le scelte dei giocatori e determinare il vincitore della manche.
Nonostante abbiano approcci differenti nella descrizione hardware, sia utilizzando SIS a livello di gate che scrivendo codice Verilog in uno stile behavioural, il datapath presenta un comportamento coerente e equivalente in entrambe le rappresentazioni.

\begin{lstlisting}[firstnumber=34]
  always @(posedge clk) begin : UPDATE_STATE
    STATO = STATO_PROSSIMO;
  end;

\end{lstlisting}


\vspace{20pt}

Il confronto delle mosse e la verifica della loro validità, per esempio, coinvolgono principalmente operazioni logiche e confronti diretti tra i dati di input.
Queste operazioni sono più naturalmente implementate in un sistema combinatorio, dove le uscite dipendono solo dagli ingressi attuali senza considerare uno stato interno.\footnote{Si noti che nonostante nel Datapath vengano elaborati direttamente gli input senza tener conto di un eventuale stato del sistema, questo non implica l'assenza di registri funzionali al confronto con dati elaborati in cicli di clock precedenti al calcolo (come per esempio il confronto di una mossa vincente con quella dello stesso giocatore nella manche precedente).}
La progettazione a \textit{gate level} per quanto riguarda la simulazione in SIS, ci ha permesso di gestire il flusso dei dati in maniera puntuale e precisa permettendoci di testare ogni blocco di porte logiche singolarmente per poi unirlo in un sistema più complesso. 


\end{document}
